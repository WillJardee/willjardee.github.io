%%%%%%%%%%%%%%%%%%%%%%%%%%%%%%%%%%%%%%%%%%%%%%%%%%%%%%%%%%%%%%%%%%%%%%%%%%%%%%%%
% Medium Length Graduate Curriculum Vitae
% LaTeX Template
% Version 1.2 (3/28/15)
%
% This template has been downloaded from:
% http://www.LaTeXTemplates.com
%
% Original author:
% Rensselaer Polytechnic Institute 
% (http://www.rpi.edu/dept/arc/training/latex/resumes/)
%
% Modified by:
% Daniel L Marks <xleafr@gmail.com> 3/28/2015
%
% Important note:
% This template requires the res.cls file to be in the same directory as the
% .tex file. The res.cls file provides the resume style used for structuring the
% document.
%
%%%%%%%%%%%%%%%%%%%%%%%%%%%%%%%%%%%%%%%%%%%%%%%%%%%%%%%%%%%%%%%%%%%%%%%%%%%%%%%%

%-------------------------------------------------------------------------------
%	PACKAGES AND OTHER DOCUMENT CONFIGURATIONS
%-------------------------------------------------------------------------------

%%%%%%%%%%%%%%%%%%%%%%%%%%%%%%%%%%%%%%%%%%%%%%%%%%%%%%%%%%%%%%%%%%%%%%%%%%%%%%%%
% You can have multiple style options; legal options are:
%
%   centered:	the name and address are centered at the top of the page 
%				(default)
%
%   line:		the name is left with a horizontal line, then the address to
%				the right
%1
%   overlapped:	the section titles overlap the body text (default)
%
%   margin:		the section titles are to the left of the body text
%		
%   11pt:		use 11 point fonts instead of 10 point fonts
%
%   12pt:		use 12 point fonts instead of 10 point fonts
%
%%%%%%%%%%%%%%%%%%%%%%%%%%%%%%%%%%%%%%%%%%%%%%%%%%%%%%%%%%%%%%%%%%%%%%%%%%%%%%%%

\documentclass[margin]{res}  

%Default font is the Helvetica postscript font
%\usepackage{helvet}

% Increase text height
\textheight=700pt
\usepackage{fancyhdr}
\pagestyle{fancy}
\renewcommand{\headrulewidth}{0pt}
\fancyhf{}
\rfoot{Page \thepage}

%\usepackage{fontspec}
% Use this line to change the font of your file. If you have the Open Dyslexic v. 1 downloaded (bold-italic- bolditalics included), you can use this line. This font addresses Dyslexic people. 
%\setmainfont{OpenDyslexic}
% Use this line to change the font of your file. If you have the Atkinson Hyperlegible downloaded (bold-italic- bolditalics included), you can use this line. This font addresses sight-impaired peoples
%\setmainfont{Atkinson Hyperlegible}

\usepackage{inputenc}


\usepackage{hyperref}
\hypersetup{
    colorlinks=true,
    linkcolor=blue,
    filecolor=magenta,      
    urlcolor=blue,
    pdftitle={Jardee Curriculum Vitae},
    }
\urlstyle{same}

\begin{document}


%-------------------------------------------------------------------------------
%	NAME AND ADDRESS SECTION
%-------------------------------------------------------------------------------
\name{William V. Jardee \vspace{3ex}}

% Note that addresses can be used for other contact information:
% -phone numbers
% -email addresses
% -linked-in profile

\address{\href{https://willjardee.github.io/}{WillJardee.github.io}\\\href{https://github.com/WillJardee}{Github.com/WillJardee}}
\address{willjardee@gmail.com\\ (406) 836-2338}

% Uncomment to add a third address
%\address{Address 3 line 1\\Address 3 line 2\\Address 3 line 3}
%-------------------------------------------------------------------------------

\begin{resume}


%-------------------------------------------------------------------------------
%	EDUCATION SECTION
%-------------------------------------------------------------------------------
\section{EDUCATION}
\raggedright
\textbf{Ph.D. in Computer Science} \hfill \textit{Aug~2022~-~Present}\\
\textit{Montana State University, Bozeman, MT} \hfill GPA: 3.81/4.0\vspace*{1ex}\\
%---
\textbf{B.S. in Physics} \hfill \textit{Aug~2018~-~May~2022}\\
\textit{Montana State University, Bozeman, MT} \hfill GPA: 3.81/4.0\\
	\hspace{3ex} Summa Cum Laude\\
	\hspace{3ex} Honors Highest Distinction\\
	\hspace{3ex} Phi Kappa Phi Honors Society\\
	\hspace{3ex} Minors in Computer Science and Mathematics


%%-------------------------------------------------------------------------------
%%	PROJECTS SECTION
%%-------------------------------------------------------------------------------
%\section{PROJECTS}
%\raggedright
%\par
%\textbf{Relativistic Runaway Electrons and Lightning Discharge; A Qualitative Overview}: 
%A paper on the building blocks of the RREA Theory, alongside motivations, computational and experimental evidence. Survey of step leaders and the related TGF emissions.
%\par
%\textbf{RREA Propagation Theory; A Theoretical and Computational Overview}:
%A delve into the theoretical derivation of RREA theory and the implementation of complete Monte Carlo simulations of particle propagation in storm-clouds. ({\sl Ongoing project})
%\par 
%\textbf{Introduction to Computational Physics}:
%An overview of Python, LaTeX , and other essential tools to computational sciences. The overview covers both fundamental concepts and detailed delves into specific topics. ({\sl Ongoing project})
%
%%-------------------------------------------------------------------------------

%-------------------------------------------------------------------------------
%	COMPUTER SKILLS SECTION
%-------------------------------------------------------------------------------
\section{\uppercase{Interests}}
\textbf{Research}\\
	\hspace{3ex} Mathematical modeling of swarm intelligence models and their hyperparameters\\
	\hspace{3ex} Ethical artificial intelligence\\
	\hspace{3ex} Explainable and interpretable artificial intelligence\\
\textbf{Teaching}\\
	\hspace{3ex} Intuitive explanations of mathematical concepts\\
	\hspace{3ex} Accessibility to mathematics for disadvantaged groups\\
\textbf{Accessibility}\\
	\hspace{3ex} Principles of effective digital accessibility\\
	\hspace{3ex} Ease of implementation of accessibility into already established workflows\\

%-------------------------------------------------------------------------------
%	COMPUTER SKILLS SECTION
%-------------------------------------------------------------------------------

\section{\uppercase{Relevant Courses and Technical \\Skills}}
{
\textbf{Languages}\\
	\hspace{3ex} Coding: Python, Java, C/C++, BASH\\
	\hspace{3ex} Mathematical Analysis: Matlab, Mathematica, Excel\\
	\hspace{3ex} Communication: Git/GitHub, LaTeX, HTML, CSS, Markdown\\
	\hspace{3ex} Computer Systems: Arch Linux, Ubuntu, Windows, Arduino\vspace*{1ex}\\
%---
\textbf{Computer Science} \\
	\hspace{3ex} Adv. AI/ML \textit{(CSCI 446, CSCI 547, CSCI 546, CSCI 550, M 508)}
		\footnote{Lables correspond with course numbers from Montana State University. CSCI: Computer Science, PHSX: Physics, M: Mathematics, STAT: Statistics.}\\
	% \hspace{3ex} Adv. Datamining \textit{(CSCI 550)}\\
	% \hspace{3ex} Adv. Artificial Intelligence \textit{(CSCI 546)}\\
	\hspace{3ex} Computational Geometry \textit{(CSCI 534)}\\
	% \hspace{3ex} Machine Learning \textit{(CSCI 547)}\\
	% \hspace{3ex} Artificial Intelligence \textit{(CSCI 446)}\vspace*{1ex}\\
	\hspace{3ex} Computation Theory \textit{(CSCI 538, CSCI 532)}\vspace{3ex}\\
%---
\textbf{Mathematics}\\ 
	\hspace{3ex} Probability Theory \textit{(PHSX 446, STAT 501)}\\
	\hspace{3ex} Analytic and Approximate Differential Equations\\
	\hspace{3ex} Operator Algebra/Metric Calculus\\
	\hspace{3ex} Linear Algebra \textit{(M 333)}\\
	\hspace{3ex} Dynamical/Chaotic Systems \textit{(M 454, M 455)}\\
	\hspace{3ex} Index/Einstein Notation\vspace*{1ex}\\
%---
\textbf{Physics}\\
	\hspace{3ex} Intro to General Relativity \textit{(PHSX 491)}\\
	% \hspace{3ex} Observational Astronomy \textit{(PHSX 491)}\\
	\hspace{3ex} Quantum Mechanics \textit{(PHSX 461, PHSX 462)}\\
	\hspace{3ex} Elementary Particle Physics \textit{(PHSX 451)}\vspace*{1ex}\\
	% \hspace{3ex} Statistical Mechanics \textit{(PHSX 446)}\vspace*{1ex}\\
%---
\textbf{Communication and Leadership}\\
	\hspace{3ex} Seminar: Worldbuilding \textit{(HONR 494)}\\
	\hspace{3ex} Leadership for Future STEM Professionals \textit{(HONR 491)}\\
}



%-------------------------------------------------------------------------------

%-------------------------------------------------------------------------------
%	EXPERIENCE SECTION
%-------------------------------------------------------------------------------

\section{\uppercase{Teaching Experience}}
%---
\textbf{Introduction to ML Grading Assistent (EN605.649)}\hfill
{\sl Jan~2023~-~Present}\\
{\sl Whitney School of Engineering; JHU, Maryland}\vspace*{1ex}\\
%---
\textbf{Course Redesign: Introduction to ML (EN605.649)}\hfill
{\sl Dec~2023~-~Jan~2024}\\
{\sl Whitney School of Engineering; JHU, Maryland}\vspace*{1ex}\\
%---
\textbf{AI Substitute Lecturer (Ethical AI)}\hfill
{\sl Nov~2023}\\
{\sl Gianforte School of Computing; MSU, Bozeman}\vspace*{1ex}\\
%---
\textbf{Hillman Scholars Tutor}\hfill
{\sl Jul~2021~-~May~2022}\\
{\sl Allen Yarnell Center for Student Success; MSU, Bozeman}\vspace*{1ex}\\
%---
\textbf{Math Stats Center Tutor}\hfill
{\sl Aug~2021~-~May~2022}\\
{\sl Mathematics Dept.; MSU, Bozeman}\vspace*{1ex}\\
%---
\textbf{Introductory Physics Proctor/Grader (PHSX 207)} \hfill 
{\sl Jan~2021~-~May~2021}\\
{\sl Physics Dept.; MSU, Bozeman}\vspace*{1ex}\\
%---
\textbf{Introductory Physics Student Lab Assistant (PHSX 205)}\hfill
{\sl Aug~2020~-~Nov~2020}\\
{\sl Physics Dept.; MSU, Bozeman}\vspace*{1ex}\\
%---
\noindent
\raggedright
\textbf{Smarty Cats Tutor}\hfill
{\sl Aug~2019~-~May~2020}\\
{\sl Allen Yarnell Center for Student Success; MSU, Bozeman}\vspace*{1ex}\\
%---
\noindent
\raggedright
\textbf{Volunteer STEM Tutor}\hfill
{\sl Oct~2019~-~Mar~2020}\\
{\sl The Rock Youth Center; Bozeman, MT}


%-------------------------------------------------------------------------------

\section{\uppercase{Research\\Experience}}
\noindent
\raggedright
\textbf{Graduate Researcher}\\
{\sl Numerical Intelligent Systems Laboratory; MSU, Bozeman}\\
\hspace{2ex} {\sl Sensors, ML, and AI in Real Time Fire Science} \hfill {\sl Spring~2024~-~present}\\
\hspace{2ex} {\sl AI/ML Exhibit at American Computer and Robotics Museum} \hfill {\sl Fall~2023~-~present}\\
\hspace{2ex} {\sl Modeling of Emergent Behavior in Ant Colony Optimmization} \hfill {\sl Fall~2022~-~present}\\
\hspace{2ex} {\sl Using CNNs and PIFs for classifying Prostate Cancer} \hfill {\sl Summer~2023}\\
\hspace{2ex} {\sl Fault Diagnosis of Fighter Planes using CTBN} \hfill {\sl Summer~2023} \vspace*{1ex}\\
%---
\noindent
\raggedright
\textbf{Undergraduate Researcher}\hfill 
{\sl Aug~2020~-~Dec~2020}\\
{\sl Dr. John Sample's Lab; MSU, Bozeman}\vspace*{1ex}\\
%---
\noindent
\raggedright
\textbf{Undergraduate Researcher}\hfill
{\sl Jan~2020~-~Apr~2020}\\
{\sl Dr. Rufus Cone's Lab; MSU, Bozeman}

%-------------------------------------------------------------------------------

\section{\uppercase{Misc.\\Experience}}
\textbf{\textbf{SPS Treasurer}}\hfill
{\sl Feb~2020~-~Jan~2022}\\
{\sl Society of Physics Students at Montana State University, Bozeman}\vspace*{1ex}\\
%---
%\textbf{Advanced Physics Lab}\hfill
%{\sl Aug~2021~-~Dec~2021}\\
%{\sl Instructed Course at MSU, Bozeman}

%-------------------------------------------------------------------------------
%	Awards
%-------------------------------------------------------------------------------
\section{\uppercase{Awards and Grants}}
\textbf{Center for Science, Technology, Ethics, and Society}\hfill {\sl Fall~2023}\vspace*{1ex}\\
\textbf{Benamin Fellowship}\hfill {\sl Sept~2022}\vspace*{1ex}\\
\textbf{Dept. Physics Outstanding Graduating Senior} \hfill {\sl Apr~2022}\vspace*{1ex}\\
\textbf{Physics Departmental Scholarship}\\
\hspace{2ex} {\sl Norman Mac Rugheimer Scholarship} \hfill {\sl Aug~2021,~Jan~2022}\\
\hspace{2ex} {\sl Asbridge Physics Scholarship} \hfill {\sl Aug 2020}\vspace*{1ex}\\
\textbf{Montana University Systems Scholarship}\hfill\hfil {\sl May 2018}\vspace*{1ex}\\
\textbf{Bertha Feaster Scholarship} \hfill {\sl May 2018}

%-------------------------------------------------------------------------------
%	Presentations
%-------------------------------------------------------------------------------
\section{\uppercase{Posters and Presentations}}

\textbf{MSU Relativity and Astrophysics (RelAstro) Seminar}\\
\hspace{3ex} {\sl Introduction to Data Exploration with Machine Learning}\hfill {\sl Nov~2022}\vspace*{1ex}\\
%---
\textbf{MSU Guest Lecturer}\\
\hspace{3ex} {\sl Introduction to Python Seminar: Building Neural Networks}\hfill {\sl Nov~2022}\vspace*{1ex}\\
%---
\textbf{MSU Student Research Celebration}\\
\hspace{3ex} {\sl Rule Extraction from a Random Forest} \hfill {\sl May~2022}\vspace*{1ex}\\
%---
\textbf{SPS Undergraduate Colloquium}\\
\hspace{3ex} {\sl How to Teach Yourself to Code} \hfill {\sl Nov~2022}\\
\hspace{3ex} {\sl RREA Propagation Theory} \hfill {\sl Oct~2021}\\
\hspace{3ex} {\sl The Better Poster Design} \hfill {\sl Feb~2021}\\
\hspace{3ex} {\sl Teaching Yourself Computer Languages} \hfill {\sl Feb~2021}\\
\hspace{3ex} {\sl Introduction to Python} \hfill {\sl Feb~2021}\\
\hspace{3ex} {\sl The Basics of Climate Physics} \hfill {\sl Sept~2020}

	
%-------------------------------------------------------------------------------
%	Outreach
%-------------------------------------------------------------------------------
\section{\uppercase{Outreach}}
\textbf{Science Mentorship Institute (sci-MI)}\\
\hspace{3ex} {\sl CS Curriculum Team}\hfill {\sl Jan~2024~-~Present}\\
\textbf{Montana Science Center}\\
\hspace{3ex} {\sl Summer Camp; Volunteer Counselor}\hfill {\sl Jun~2022~-~Jul~2022}\\
\hspace{3ex} {\sl Science After Dark; Event Volunteer}\hfill {\sl Oct~2022}\\
\hspace{3ex} {\sl Pride in STEM; Event Volunteer}\hfill {\sl Nov~2022}\vspace*{1ex}\\
\textbf{Museum of the Rockies}\\
\hspace{3ex} {\sl Grossology; Event Volunteer}\hfill {\sl Oct~2021}\vspace*{1ex}\\
\textbf{Society of Physics Students}\vspace*{0.5ex}\\
\hspace{3ex} {\sl Liquid Nitrogen Ice Cream; Organizer}\hfill {\sl Oct~2021}\vspace*{0.5ex}\\
\hspace{3ex} {\sl Careers in Industry Panel; Moderator}\hfill {\sl ~Mar~2021,~Oct~2020}


%%-------------------------------------------------------------------------------
%%	Interests
%%-------------------------------------------------------------------------------
%\section{INTERESTS}
%Science communication, computational physics, particle physics, chaotic systems, RREA propagation theory
%
%%-------------------------------------------------------------------------------
\end{resume}

%\vfill
%\fbox{\parbox{.95\linewidth}{Accessibility should be the default, not an afterthought. For this reason, I have chosen to use \href{https://brailleinstitute.org/freefont}{Atkinson Hyperlegible} as the font for this document. The font was specially designed to help people with low vision and has undergone many iterations, always improving the experience for people with vision impairment and people who enjoy pretty fonts. For people who want to look into helping people with processing disorders like dyslexia, look into the research-driven, open-source project \href{https://opendyslexic.org/}{Open Dyslexic}.}}
\end{document}